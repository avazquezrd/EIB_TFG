% ------------------------------------------------------------ %
% TIPO Y PAQUETES                                              %
% ------------------------------------------------------------ %

\documentclass[11pt, a4paper]{report}
\usepackage[spanish, es-tabla, activeacute]{babel}
% \usepackage[utf8]{inputenc}
\usepackage{titlesec}
\usepackage{hyperref}
\usepackage{graphicx}
\usepackage{geometry}
\usepackage{indentfirst}
\usepackage{xcolor}
\usepackage{verbatim}
\usepackage{keyval}
\usepackage{etoolbox}
\usepackage{fontspec}
\usepackage{titling}
\usepackage{tikz}
\usepackage{caption}
\usetikzlibrary{positioning}


% ------------------------------------------------------------ %
% PARTICIÓN DE PALABRAS AL FINAL DE LÍNEA                      %
% ------------------------------------------------------------ %

\pretolerance=2000
\tolerance=3000

% ------------------------------------------------------------ %
% FUENTES DEL DOCUMENTO                                        %
% ------------------------------------------------------------ %

 \setmainfont[
 Path          = fonts/,
 Ligatures     = TeX,
 UprightFont   = EHUSans-Light.otf,
 BoldItalicFont= EHUSans-BoldItalic.otf,
 BoldFont      = EHUSans-Bold.otf,
 ItalicFont    = EHUSans-Italic.otf
 ]{EHUSans}

\newfontfamily{\sansthick}[
 Path          = fonts/,
 BoldFont      = EHUSans-Bold.otf,
 UprightFont   = EHUSans-Regular.otf
 ]{EHUSans}
 
 \newfontfamily{\sansblack}[
 Path          = fonts/,
 UprightFont   = EHUSans-Black.otf
 ]{EHUSans}
 
 \newfontfamily{\serif}[
 Path          = fonts/,
 Ligatures     = TeX,
 UprightFont   = EHUSerif-Light.otf,
 BoldItalicFont= EHUSerif-BoldItalic.otf,
 BoldFont      = EHUSerif-Bold.otf,
 ItalicFont    = EHUSerif-Italic.otf
 ]{EHUSerif}
  
 \newfontfamily{\serifthick}[
 Path          = fonts/,
 BoldFont      = EHUSerif-Bold.otf,
 UprightFont   = EHUSerif-Regular.otf
 ]{EHUSerif}
 
 \newfontfamily{\serifblack}[
 Path          = fonts/,
 UprightFont   = EHUSerif-Black.otf
 ]{EHUSerif}
 
% ------------------------------------------------------------ %
% INSERTAR ETIQUETA DEBAJO DE LAS TABLAS					   %
% ------------------------------------------------------------ %

 
% ------------------------------------------------------------ %
% COMANDO PARA CENTRAR FIGURAS DIRECTAMENTE                    %
% \im[width, caption]{path}				                       %
% ------------------------------------------------------------ %

\DeclareCaptionFont{ehu}{\fontsize{9}{10}\mdseries}
\usepackage[font={ehu}, labelfont={bf}]{caption}
\captionsetup[table]{position=bottom}

\makeatletter

\newlength{\im@width}
\define@key{im}{width}{\setlength\im@width{#1}}
\define@key{im}{caption}{\def\im@caption{#1}}

\setkeys{im}{width=\linewidth, caption={}}

\newcommand{\im}[2][]{
	\setkeys{im}{#1}
	\ifdefempty{\im@caption}{
		\begin{figure}[h!]
			\centering
			\includegraphics[width=\im@width]{#2}
			{\caption{}}
		\end{figure}
	}{
		\begin{figure}[h!]
			\centering
			\includegraphics[width=\im@width]{#2}
			\caption{\im@caption}
	\end{figure}}
	\setkeys{im}{width=\linewidth, caption={}}
}

\makeatother

% ------------------------------------------------------------ %
% AÑADIR CUARTO NIVEL DE TÍTULO                                %
% ------------------------------------------------------------ %

\setcounter{secnumdepth}{4}
\titleformat{\paragraph}
{\normalfont\normalsize\bfseries}{\theparagraph}{1em}{}
\titlespacing*{\paragraph}
{0pt}{3.25ex plus 1ex minus .2ex}{1.5ex plus .2ex}

% ------------------------------------------------------------ %
% ELIMINAR CABECERA DE CAPÍTULOS                               %
% ------------------------------------------------------------ %

\titleformat{\chapter}{\normalfont\Huge\bfseries}{}{0pt}{\Huge}
\makeatletter
\let\latexl@chapter\l@chapter
\def\l@chapter#1#2{\begingroup\let\numberline\@gobble\latexl@chapter{#1}{#2}\endgroup}
\makeatother

% ------------------------------------------------------------ %
% MÁRGENES Y DISTANCIA ENTRE PÁRRAFOS						   %
% ------------------------------------------------------------ %

\geometry{top=2.5cm,left=3cm,right=3cm,bottom=2.5cm}
\setlength{\parskip}{8pt}

% ------------------------------------------------------------ %
% TÍTULO DEL ÍNDICE                                            %
% ------------------------------------------------------------ %

\addto\captionsspanish{
	\renewcommand{\contentsname}{Índice}
	\renewcommand{\listtablename}{Lista de tablas}
	\renewcommand{\listfigurename}{Lista de figuras}
}

% ------------------------------------------------------------ %
% DIRECTORIO DE IMÁGENES                                       %
% ------------------------------------------------------------ %

\graphicspath{{images/}}

% ------------------------------------------------------------ %
% FORMATO DEL TÍTULO                                           %
% El documento principal debe incluir este código:             %
%% Título:
%\newcommand{\titulo}{<Título del trabajo>}
%% Curso académico:
%\newcommand{\curso}{<XXXX-XXXX>} 
%% Carrera:
%\newcommand{\carrera}{Grado en XXXXXXXX} 
%% Alumno/Alumna:
%\newcommand{\alumno}{<apellido 1, apellido 2, nombre>}
%% Dirección 1 del TFG:
%\newcommand{\direccionA}{<apellido 1, apellido 2, nombre>}
%% Dirección 2 del TFG (si no la hay, comentar la línea aquí y también en preamble.tex):
%\newcommand{\direccionB}{<apellido 1, apellido 2, nombre>}
%% Fecha:
%\newcommand{\fecha}{<XXXX, día, mes, año>}
% ------------------------------------------------------------ %

\newcommand{\portada}{
	{\raggedleft\includegraphics[width=8cm]{corp/ehu}}
	\\[3\baselineskip]
	\begin{center}
		\LARGE{\MakeUppercase{\carrera}}
		
		\Huge{\textbf{\MakeUppercase{Trabajo de fin de grado}}}
		\\[1\baselineskip]
		\begin{tikzpicture}
			\renewcommand{\baselinestretch}{0.75}
			\node (A) at (0.03cm,-0.03cm) [inner sep=20pt, draw, minimum width=\linewidth, text width=13cm, minimum height=104pt, fill=black, align=center]
			{\huge{\MakeUppercase{\textbf{\textit{\titulo}}}}};
			\node (B) at (0,0) [inner sep=20pt, draw, minimum width=\linewidth, text width=13cm, minimum height=104pt, fill=white, align=center] 
			{\huge{\MakeUppercase{\textbf{\textit{\titulo}}}}};
		\end{tikzpicture}
	\end{center}
	\renewcommand{\baselinestretch}{1.25}
	\begin{tikzpicture}[remember picture, overlay]
		\node (C) at ([yshift=-6.95cm, xshift=9.7cm]current page.west) [inner sep=10pt, draw, thick, minimum width=13.34cm, minimum height=64pt, align=left, fill=white]{};
		\node (D) [right=0.2cm of C.west, anchor=west, align=left]
		{\small{\textbf{Alumno/Alumna: }\alumno}\\
			\small{\textbf{Director/Directora (1): }\direccionA}
			% Comentar la siguiente línea si sólo hay una direcicón del TFG. Eliminar también de la anterior el (1).
			\\\small{\textbf{Director/Directora (2): }\direccionB}
			};
	\end{tikzpicture}
	\renewcommand{\baselinestretch}{1}
	\begin{tikzpicture}[remember picture, overlay]
		\node (E) at ([yshift=-9.4cm, xshift=8.5cm]current page.west) [inner sep=10pt, draw, thick, minimum width=13.34cm, minimum height=27pt, align=left, fill=white]{};
		\node (F) [right=1.4cm of E.west, anchor=west]
		{\small{\textbf{Curso: }\curso}};
	\end{tikzpicture}
	\begin{tikzpicture}[remember picture, overlay]
		\node (G) at ([yshift=-11.6cm, xshift=13.4cm]current page.west) [inner sep=10pt, draw, thick, minimum width=7.93cm, minimum height=27pt, align=left, fill=white]{};
		\node (H) [left=0.2cm of G.east, anchor=east]
		{\small{\textbf{Fecha: }\fecha}};
	\end{tikzpicture}
}
